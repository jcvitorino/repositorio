\chapter{PROCEDIMENTOS METODOLÓGICOS/MÉTODOS E TÉCNICAS}

Para o desenvolvimento deste trabalho, serão realizadas as seguintes atividades:

\begin{itemize}
\item Levantamento bibliográfico.

O levantamento bibliográfico, ocorrerá durante todo a realização do trabalho.

\item Estudo de métodos para a predição de elementos regulatórios.

Nesta fase serão estudas técnicas de predição de elementos regulatórios, identificando as técnicas com resultados mais precisos.

\item Definição do conjunto de dados.

Neste passo, será feita a coleta e separação de sequências promotoras do DNA do genoma da soja. Para isto, serão utilizados bancos de dados publicos que contêm sequências promotoras.

\item Identificação e análise de elementos regulatórios.

Nesta etapa, será implementado um sistema de identificação e classificação dos elementos regulatórios, que são ligados com fatores de transcrição da classe DREB.

\item Validação dos elementos encontrados no genoma da soja.

Neste ponto os elementos encontrados serão comparados com elementos já existentes, avaliando a precisão dos métodos implementados.

\item Redação do TCC.

O TCC será redigido durante todo a elaboração do trabalho.

\end{itemize}

