\chapter{RESUMO}

<<<<<<< .mine
Nos últimos anos muitos estudos, estão sendo realizados para entender a regulação de um gene. O completo entendimento desta será um grande avanço na genética. Em particular nas plantas, este entendimento poderá ajudar, em pesquisas de melhoramentos genéticos.

Para entender a regulação de um gene, que é um complexo processo que envolve diversos fatores dentro das células, estudos estão sendo feitos para a identificação de sequências específicas no DNA, chamadas de elementos regulatórios. Estes elementos, juntamente com os fatores de transcrição, são responsáveis pelo início da transcrição em uma célula. Eles funcionam como mecanismos de resposta das células a eventos interna e externamente nas células, como: mudanças hormonais, elevação da temperatura e seca. Estas mudanças muitas vezes afetam negativamente as plantas, interferindo na produtividade da mesma. Com o avanço nas descobertas dos elementos regulatórios, poderão ser desenvolvidas plantas mais resistentes a essas condições adversas. Esse projeto apresenta abordagens computacionais que serão desenvolvidas no empenho de encontrar os elementos regulatórios, que são ativados quando a planta é exposta a estresses. Os estudos serão conduzidos no genoma da soja, uma cultura amplamente utilizada e importante para a economia nacional.=======
Nos últimos anos, alguns fatores de transcrição que regulam a expressão de vários genes relacionados com o estresse foram descobertos. Esses fatores de transcrição são subdivididos em várias classes como a Dehydration Responsive Element Binding Proteins (DREB), que está relacionada com a seca e a desidratação da planta. O entendimento dos DREBs é importante para o desenvolvimento de plantas com tolerância a estresses abióticos como a seca, alta salinidade e baixa temperatura. Esse projeto apresenta abordagens computacionais que serão desenvolvidas no empenho de encontrar os elementos regulatórios na soja, que são ativados pelos fatores de transcrição pertencentes classe DREB.>>>>>>> .r3
