\chapter{Busca de genes alvos de TFs}

Para descifrar a regulação genica de um organismo, e consequentemente ter um domínio na manipulação genética do organismo, muitas metodologias \textit{in vitro} e \textit{in silico} foram propostas. Das metodologias \textit{in silico}, algumas focam na busca de genes alvos de TFs, e outras na classificação da funcionalidade de um gene desconhecido a partir de um gene conhecido. A seguir serão apresentados algumas metodologias para a classificação da funcionalidade de genes. Em especial a seção \ref{dreb} apresenta a busca de genes alvos do DREB que é o foco desta pesquisa.

 
%%%%%%%%%%%%%%%%%%%%%%%%%%%%%%%%%%%%%%%%%%%%%%%%%%%%%%%%%%%%%%%%%%%%%%%%%%%%%%%%%%%%%%%%%%%%%%%%%%%%%%%%%%%%%%%%%%%%%%%%





\section{An in silico strategy identified the target gene candidates regulated by dehydration responsive element binding proteins (DREBs) in Arabidopsis genome \cite{Wang2009}} \label{dreb}
% Como a idéia é explicar os DREBS antes aqui não vai ser explicado novamente.
Para inferir genes alvos do DREB na \textit{Arabidopsis} \cite{Wang2009}, criaram uma estratégia computacional, que combina a análise de elementos regulatórios (TFBS, do inglês \textit{Transcription factor binding site}) e aprendizado de máquina.

Eles utilizaram como conjunto de dados: sequências da região promotora (PR, do inglês \textit{Promoter region}) de genes alvos do DREB identificados experimentalmente, estes são identificados como genes mestres (MGs, do inglês \textit{Master genes}); \textit{DRE frame sequences} (DFSs), fragmentos de DNA com 206 pb (pares de bases), retirados das PRs de MGs, contendo a subsequência consenso (A/GCCGAC) que é a região onde o DREB se conecta em um gene, identificada no artigo como DRE-motif; \textit{Non-DRE frame sequences} (nDFS) fragmentos de de DNA com 206 pb, retirados das PRs de gene aleatórios, com um DRE-motif inserido artificialmente na região central. No total foram encontrados 48 DFSs de PRs de MGs, que foram considerados como dados positivos, e 1000 nDFSs como dados negativos.

Após a seleção do conjunto de dados, foi construído um classificador SVM (do inglês, \textit{support vector machine}) para categorizar DFSs e nDFSs. O vetor de características utilizado no SVM, foi um vetor contendo hexamers (pequenos fragmentos de DNA com 6 nucleotídeos), que foram selecionados através do algoritmo HexDiff \cite{Chan2005}. Este algoritmo computa as pontuações da frequência de um hexamer no conjunto positivo $F_{p}(h)$ e negativo $F_{n}(h)$, em ambos filamentos de DNA. Depois de calculada a frequência para cada hexamer em ambos os conjuntos de dados foi calculado a pontuação da razão $R(h)$, como:
\begin{equation}
R(h) = \frac{F_{p}(h)}{F_{n}(h)}
\end{equation}
Os hexamers com a pontuação da razão maior que um limite estabelecido, foram colocados em um vetor $H_{d}$, que é o vetor de características da SVM. Para cada hexamer em $H_{d}$ que pertence ao conjunto de DFS foi atribuído o rótulo (+1), para os de nDFS receberam (-1). A função Kernel utilizada no classificador SVM foi a RBF. Para o treinamento do classificador, visto que havia uma disparidade grande entre os dados positivos e negativos, foram usados apenas 100 amostras negativas e todas as 48 positivas. Já na classificação de DFSs e nDFSs foram utilizados as 1000 amostras negativas, os dados positivos foram reposicionados para que pudesse existir um sobre-amostragem, uma vez que eles estavam em um quantidade bem menor (48 DFSs), assim o conjunto de dados positivos passou para 500 amostras.

Para a previsão de genes alvos do DREB foi primeiramente, selecionados genes em todo o genoma da \textit{Arabidopsis}, cuja a PR tem a sequência consenso do DREB (A/GCCGAC). Então foram selecionadas as regiões a esquerda da sequência consenso e a direita, ambas com 100 pb, que juntamente com o consenso forma uma subsequência de 206 pb. O conjunto de dados formado, com o procedimento descrito, foi classificado com o classificador SVM. Eram considerados genes alvos do DREB, todos os gene que contem  pelo menos um DFSs em sua PR. No total, 474 genes foram preditos pela SVM e considerados fortes candidatos a serem alvos do DREB.

Segundo os autores, apenas encontrar regiões nas PRs de um gene contendo o consenso DREB, não é suficiente para inferir este gene como alvo do DREB, devido as perdas de características do DRE-motif. Vários estudos, apontam que TFBS distribuídos na PR influenciam na ligação de um TF e seu alvo. O que pode ser entendido que, na PR de genes alvos do DREB não somente o DRE-motif mas também outros TFBS, influenciam na conexão de um DREB na sequência. Estes outros TFBS agem como auxiliares para promover a conexão do DREB nos genes alvos. Com a utilização do HexDiff é selecionado regiões conservadas, que podem ser partes de TFBS que influenciam na ação do DREB, portanto um DFS será formado além da região consenso, também por sub-regiões conservadas.

%%%%%%%%%%%%%%%%%%%%%%%%%%%%%%%%%%%%%%%%%%%%%%%%%%%%%%%%%%%%%%%%%%%%%%%%%%%%%%%%%%%%%%%%%%%%%%%%%%%%%%%%%%%%%%%%%%%%%%%%%



\section{Classifying transcription factor targets and discovering relevant biological features \cite{Holloway2008}}

%Holloway2008
Neste trabalho \cite{Holloway2008}, projetaram uma metodologia, utilizando aprendizado de máquina, para a predição de genes alvos de fatores de transcrição (TF, do inglês transcription factor) específicos. Com os resultados obtidos eles puderam construir e analisar a rede regulatória do \textit{Saccharomyces cerevisiae}, uma levedura muito utilizada em estudos genéticos, por ter um genoma pequeno e mais simples comparado a outros organismos.

%definir TF no texto base
Para a classificação dos genes alvos de TFs, eles utilizaram máquinas de vetores de suporte (SVM, do inglês support vector machine). Como justificativa da utilização do SVM, os autores afirmam que: os conjuntos de dados genômicos têm uma alta dimensionalidade, detalhadamente, o conjunto pode chegar a milhares ou dezena de milhares de características numéricas para descrever um gene. Assim muitos algoritmos classificadores, podem ter um baixo desempenho com um numero alto de características, ao contrario do SVM que tem um bom desempenho com dados de alta dimensionalidade.

% Definir elementos regulatórios no texto base
Como dados de entrada positivo, para o treinamento da SVM, foram pegos genes com elementos regulatórios conhecidos, que se ligam a determinados TFs, que também são conhecidos, sabe-se da existência dos elementos regulatórios, de seus respectivos genes e dos TFs, por meio de publicações na literatura. O conjunto negativo foi pego de subconjuntos de genes que têm um alto \textit{p}-valor, consequentemente com uma probabilidade baixa de ter uma ligação aos TFs utilizados. Podemos perceber que, para cada TF em que deseja-se encontrar os genes que eles regulam é necessário a construção de um classificador (ou treinamento do classificador baseado no TF).


Para o treinamento do classificador, é recomendado que o conjunto negativo tenha pelo menos três vezes o tamanho do conjunto positivo. No conjunto negativo estão os genes que mostraram uma probabilidade baixa de serem regulados pelos TFs usados para prever os genes que eles regulam.  Depois de selecionado o conjunto negativo são construídos 50 classificadores, para cada TF, utilizando diferentes subamostras do conjunto negativo, com o mesmo tamanho da amostra positiva. É utilizado 50 classificadores para cada TF, porque genes com alto \textit{p}-valor são associados ao conjunto negativo e eles têm grandes probabilidades de não terem ligações com um dos TFs utilizados, mas há uma pequena probabilidade de um gene ser associado incorretamente ao conjunto negativo. Com os 50 classificadores este inoportuno é suavizado.


Para a avaliação de cada classificador é usado a abordagem  \textit{leave-one-out cross-validation} (LOOCV), e também as medidas de desempenho: precisão e valor preditivo positivo (PPV, do inglês positive predictive value), que são usadas como média entre os 50 classificadores.

As características do classificador são selecionadas aplicando o algoritmo SVM \textit{recursive feture elimination} (SVM-RFE), que otimiza o vetor \textbf{w} da SVM, para conter componentes altas, que são melhores para separar as classes positivas e negativas de dados. O processo de seleção de características, usando o SVM-RFE, é repetido até atingir o numero desejado de características, que é 1500, este numero é escolhido porque, quando é usado 1500 características a medida de precisão é aproximadamente 85\%, que é uma precisão consideravelmente boa. Estas características são escolhidas para cada TF e são guardadas durante a avaliação dos 50 classificadores. No final haverá um grande conjunto de características para um TF, lembrando que os elementos desse conjunto são subconjuntos de características. Então são escolhidos os 40 maiores características, de cada subconjunto de características, e são guardas em uma lista, e é contada quantas vezes cada característica apareceu. Esta lista é rearranjada, posicionando os elementos que tem uma frequência de aparição maior no topo, assim este é o conjunto final de características. Essas características inclui um conjunto diverso de dados incluindo sequências promotoras, medidas de expressão de um gene, conservação filogenética de elementos de sequências, sobre-representação de sequências promotoras, temperatura de fusão de promotores, e outras.

O esquema usado para a montagem dos classificadores funciona da seguinte maneira: passo 1, é reunido o conjunto de dados positivos e negativos, totalizando \textit{n}; passo 2, utilizando o LOOCV é pego \textit{n-1} genes para o conjunto de treinamento e 1 para o conjunto de teste; passo 3 então é usado o SVM-RFE para classificar as características no conjunto de treinamento; passo 4 construir um classificador SVM com as 1500 características. Salvar as características; passo 5, classificar o gene deixado de fora do conjunto de treinamento; passo 6 repetir os passos 2-5 até completar o LOOCV. Salvar todas as características; passo 7 calcular as estatísticas de desempenho (precisão, PPV, etc.); passo 8 repetir passos 1-5 50 vezes; último passo, calcular as estatísticas de desempenho final (média da precisão, média do PPV).

Os autores aplicaram este método em 163 TFs do \textit{S. cerevisiae}, com os resultados obtidos eles construíram uma rede de regulação que foi disponibilizada em um servidor web (http://cagt10.bu.edu/TFSVM/main.htm), onde é possível consultar quais são os genes alvos de um TF, ou quais TFs se ligam em um gene.
%%%%%%%%%%%%%%%%%%%%%%%%%%%%%%%%%%%%%%%%%%%%%%%%%%%%%%%%%%%%%%%%%%%%%%%%%%%%%%%%%%%%%%%%%%%%%%%%%%%%%%%%%%%%%%%%%%%%%%%%%




\section{Combining classifiers to predict gene function in Arabidopsis thaliana using large-scale gene expression measurements \cite{Lan2007}}

O foco deste trabalho é a classificação dos genes segundo suas funcionalidades, mais especificamente de genes que estão envolvidos na resposta das plantas a estresses. O trabalho foi conduzido por \cite{Lan2007}, e a planta usada como teste foi a \textit{Arabidopis thaliana}.

Para a classificação dos genes foram desenvolvidos cinco métodos de aprendizado supervisionado, que foram: \textit{Logistic Regression} (LR), \textit{Linear Discriminant Analysis} (LDA), \textit{Quadratic Discriminant Analysis} (QDA), \textit{Naive Bayes} (NB) e \textit{K-Nearest Neighbors} (KNN). Foram escolhidos estes métodos básicos, porque eles requerem pouca computação e os resultados são bons o suficiente para se fazer análises biológicas. Todos os classificadores montados com os métodos, retornam um valor discriminativo para cada gene avaliado. Cada gene é representado por vetor com 290 dimensões cujas as componentes são valores de expressão do gene em 290 condições experimentais. O valor retornado por um classificador tem que ser maior que o limite estabelecido.

O conjunto de dados usado foi extraído de experimentos relacionados com estresses. No total foram 22.746 genes sobre 290 condições experimentais diferentes. Desses 22.746 genes, 11.553 são genes anotados e com suas funções conhecidas, e desses 1.031 respondem à estresses. Os genes anotados formaram o conjunto de dados de treinamento, onde um gene foi considerado positivo se ele foi anotado como um gene de resposta a estresse, e negativo para os demais. Foram 11.193, o total de genes não anotados, estes foram usados para fazer as previsões. Podem haver alguns falsos negativos, visto que genes que não foram descobertas as funcionalidades, são introduzidos no conjunto negativo, entretanto eles podem ser genes de resposta a estresses.

Antes dos dados serem treinados, eles passaram por um pré-processamento, para reduzir a dimensão do conjunto de dados, para que durante o aprendizado supervisionado os dados sejam usados eficientemente. Isto foi feito com o algoritmo \textit{Principal components analysis} (PCA), que mapeia vetores de alta dimensão para um dimensão menor. O PCA aplicado ao conjunto de dados, que originalmente tinha 290 dimensões, reduziu a dimensão para as dimensões de 5, 10, 15, 20, 40 e 100.

Os classificadores são treinados com todas as dimensões geradas pelo PCA e pela dimensão original com exceção do KNN, no caso do KNN é usado diferentes valores de K na dimensão original. Então é escolhida a dimensão em que cada algoritmo obteve melhores resultados (no caso do KNN o melhor K). Os classificadores com a sua melhor dimensão, são combinados em um só classificador, onde o valor de discriminação do classificador combinado é uma combinação linear dos discriminastes dos classificadores individuais. Como esperado o classificador combinado obteve melhores resultados que os classificadores individuais.

O resultado final da classificação é um conjunto de genes, que podem ser posicionados quanto ao valor discriminante do gene responder a estresses, ficando os com maiores valores no topo.
%%%%%%%%%%%%%%%%%%%%%%%%%%%%%%%%%%%%%%%%%%%%%%%%%%%%%%%%%%%%%%%%%%%%%%%%%%%%%%%%%%%%%%%%%%%%%%%%%%%%%%%%%%%%%%%%%%%%%%%%%




\section{Using hexamers to predict cis-regulatory motifs in Drosophila \cite{Chan2005}}

Neste trabalho \cite{Chan2005} desenvolveu o algoritmo HexDiff. Este algoritmo busca agrupações de TFBS, que atuam juntos na regulação de um gene. Um agrupamento de TFBS é comumente conhecido por CRM (do inglês \textit{cis-regulatory modules}).

O HexDiff é um tipo de algoritmo de aprendizado de máquina, e foi projetado para discriminar dois tipos de sequências de DNA: CRM, e non-CRM(não agrupamento de TFBS). Para fazer a classificação é necessário um conjunto de dados de treinamentos, que é obtido através de conhecidos CRMs, que são colocados no conjunto positivo de treinamento, os não conhecidos, os non-CRMs, são inseridos no conjunto negativo. Os dados para teste foram pegos de {citar}, de conhecidos CRMs da  \textit{Drosophila}. Foram encontrados 16 genes que, contém um total de 52 CRMs. Após a seleção dos dados é calculada a frequência de cada hexamer (subsequência de nucleotídeos de 6 pb), no conjunto negativo $f_{p}(h)$ e positivo $f_{n}(h)$, para calcular a razão $R(h)$:
\begin{equation}
R(h) = \frac{F_{p}(h)}{F_{n}(h)}
\end{equation}
Os hexamers que obtiverem um alto valor de $R(h)$, são colocados no conjunto $H_{d}$. Com isto $H_{d}$ vai ter hexamers que são mais comuns em CRMs do que em  non-CRMs.

Depois de gerado o conjunto $H_{d}$, ele é usado para classificar cada posição em uma sequência não conhecida como uma sequência CRM e non-CRM. Para fazer a classificação é construído uma janela na sequência, entre 1000-2000 pb, que a cada rodada é movida 1 pb na sequência, e é calculado a pontuação $S_{i}$ para cada posição $i$ da janela na sequência, pelo produto da razão $R(h)$ e o numero de aparições de um hexamer $n(h_{d})$ em $H_{d}$ na janela, qualquer posição que exceder o limite é considerada um CRM.

Para a avaliação foi usado a abordagem  \textit{leave-one-out cross-validation} (LOOCV), onde dos 16 genes encontrados 15 são treinados e 1 é usado como teste, este processo é feito até que todos os 16 genes sejam usados no conjunto de teste. A precisão das previsões do modelo foi medida com a correlação de Matthew. 

Quando aplicado a no-CRMs da \textit{Drosophila}, além dos CRMs já conhecidos serem previstos, outros 10 CRMs foram encontrados e indicados como fortes candidato a CRMs.
%%%%%%%%%%%%%%%%%%%%%%%%%%%%%%%%%%%%%%%%%%%%%%%%%%%%%%%%%%%%%%%%%%%%%%%%%%%%%%%%%%%%%%%%%%%%%%%%%%%%%%%%%%%%%%%%%%%%%%%%%





\section{Cis-regulatory element based targeted gene finding: genome-wide identification of abscisic acid- and abiotic stress-responsive genes in Arabidopsis thaliana \cite{Zhang2005}}

Neste trabalho os autores, projetaram uma aplicação para para encontrar genes alvos de fatores de transcrição, na \textit{Arabidopsis thaliana} ,que são induzidos por ácido abscísico(ABA, do inglês \textit{abscisic acid})e por estresses abióticos.

Os dados utilizados foram coletados de plantação de \textit{A.thaliana}, que cresceu a uma temperatura de 24ºC durante 10 dias, algumas mudas foram expostas ao ABA e/ou estresses abióticos, e outras não. Das sequências extraídas do experimento, foram coletados no total 366 regiões promotoras de genes regulados após a exposição ao ABA e/ou estresses abióticos para análises.

Para encontrar os genes alvos, primeiramente foi calculado a pontuação de cada subsequência de tamanho $w$ em uma determinada sequência, baseado em padrões com relevâncias biológicas (\textit{motifs}) e um modelo de \textit{background}. Um \textit{motif} $W$ de tamanho $w$ é representado por uma matriz de peso (PMW, do inglês \textit{Position Weight Matrix}), com $PWM \Theta_{W} = (q_{l,b})$, onde $(q_{l,b})$ é a probabilidade de encontrar a base $b$ na posição $i$ do \textit{motif}, já o modelo de \textit{background} é criado utilizando o modelo de Markov. O modelo de 
\textit{background} calcula a probabilidade de uma base começar na $j-$ésima posição com $P(j|B_{m}) = \prod_{l=1}^wP(b_{j_l-1}|b_{j+l-2}...b_{j+l-l})$, onde $B_{m}$ é a $m-$ésima ordem do modelo de Markov, e $b_{j}$ é a $j-$ésima base da sequência. Também é calculada a probabilidade de uma subsequência ser um \textit{motif} $\Theta_{W}$, utilizando também o modelo de Markov, com $P(j|\Theta_{W}) = \prod_{l=1}^w q_{l,b_{j+l-1}}$, onde $q_{l,b_{j+l-1}}$ é a probabilidade de encontrar a base $b_{j+l-1}$ na posição $l$-ésima de um \textit{motif}. Então foi calculada a \textit{log-ratio} $A_{j,\Theta_{W},B_{m}} = ln\frac{P(j|\Theta)_{W}}{P(j|B_{m})}.$ A pontuação de uma sequência $S$ é feita utilizando dois \textit{motifs} $\Theta_{M}$ e $\Theta_{N}$ , neste caso um deles é o ABRE que é o elemento regulatório alvo, quando uma planta é exposta ao ABE, são consideradas todas as posições $i$ e $j$ dentro de $S$ e a combinação da pontuação das posições é computada como $A_{S,\Theta_{M},\Theta_{N},B_{m}} = max_{i,j}(A_{i,\Theta_{M},B_{m}} + A_{j,\Theta_{N},B_{m}})$. A combinação com maior pontuação é atribuída para a sequência, genes com sua sequência com maior pontuação têm uma probabilidade maior de serem genes alvos.

Como resultados foram encontrados entre 1825 genes induzidos a estresses, 1530 onde pelo menos uma funcionalidade poderia ser atribuída. E foram selecionados 150 com maior pontuação onde 126 estão classificados em alguma categoria funcionalidade. O que levou aos autores a concluir que podem existir muitas atividades de regulações genicas após a exposição ao ABA.

%%%%%%%%%%%%%%%%%%%%%%%%%%%%%%%%%%%%%%%%%%%%%%%%%%%%%%%%%%%%%%%%%%%%%%%%%%%%%%%%%%%%%%%%%%%%%%%%%%%%%%%%%%%%%%%%%%%%%%%%%





\section{CisOrtho: A program pipeline for genome-wide identification of transcription factor target genes using phylogenetic footprinting \cite{Bigelow2004CisOrtho}}

CisOrtho é software desenvolvido por \cite{Bigelow2004CisOrtho}, que identifica alvos de fatores de transcrição com um específico elemento regulatório definido, utilizando rastros filogenéticos. O programa foi usado nos genomas de dois invertebrados, o \textit{Caenorhabditis elegans} e o \textit{Caenorhabditis briggsae}.

O primeiro passo é, identificar, classificar, e associar as regiões não codificantes dos genes, isto é feito com as informações contidas em arquivos GFF (do inglês, \textit{General Feture Format}). Foram retiradas as regiões codificantes porque é improvável que exita elementos regulatórios nestas regiões e por elas serem improprias para buscas filogenéticas apresentando uma alta conservação. O segundo passo consiste em construir uma PWM (do inglês, \textit{Position Weight Matrix}) para um conjunto de elementos regulatórios definidos experimentalmente, neste trabalho foram usados os elementos regulatórios dos fatores de transcrição TTX-3 e CEH-10. Para a construção da PWM foi usado o software HMMER \cite{Eddy1998}, este software utiliza o modelo oculto de Markov para gerar a PWM. A PWM resultante é usada como entrada no CisOrtho, que faz uma busca nas sequências não codificantes, atribuindo uma pontuação para cada bloco de subsequência de tamanho $n$ (chamado de janela), onde o objetivo é encontrar subsequências com a maior pontuação $N$, e com no máximo $D$ subsequências, onde $N$ e $D$ são definidos pelo usuário. No quarto passo é feita a análise filogenética, são utilizados conjuntos de mapeamentos ortólogos\footnote{Sequências conservadas de diferentes organismo, que tem o mesmo ancestral} entre \textit{C. elegans} e \textit{C. briggasae} que prove pares de subsequências ortólogas. Os pares de subsequências que apresentam uma alta pontuação são guardados, como um par de subsequência. O ultimo passo classifica os pares de subsequências  de acordo com as suas pontuações e \textit{mismatches}\footnote{Bases diferentes, em uma mesma posição, neste caso é comparado as bases do \textit{C. elegans} e \textit{C. briggasae}}.

Com o CisOrtho foi possível identificar novos genes alvos dos fatores de transcrição TTX-3 e CEH-10, foram encontrados 14 genes com subsequências com alta pontuação que satisfaziam os critérios para serem alvos de TTX-3/CEH-10, também foram encontrados genes com baixa pontuação, mas que atendiam os critérios para serem alvos de TTX-3/CEH-10, um total de 11 genes. Para subsequência com uma grande conservação mas, com uma baixa pontuação, foram feitas análises e foram encontrados também genes alvos.




%%%%%%%%%%%%%%%%%%%%%%%%%%%%%%%%%%%%%%%%%%%%%%%%%%%%%%%%%%%%%%%%%%%%%%%%%%%%%%%%%%%%%%%%%%%%%%%%%%%%%%%%%%%%%%%%%%%%%%%%%
\section{Tabela comparativa}


\begin{tabularx}{\textwidth}{ |X|X|X|X|X| }    \hline
		  Trabalho	   & Organismo    & Entradas     & Técnicas usadas & Resultados obtidos       \\ \hline \hline	

    \cite{Wang2009}     	 &  \textit{Arabidopsis} & sequências promotoras de genes que contém o DRE-motif e sequências que não contém o DRE-motif  & O algoritmo HexDiff e SVM & 474 genes alvos  \\ \hline
    \cite{Holloway2008} 	   & \textit{Saccharomyces cerevisiae}  & sequências promotoras; medidas de expressão de um gene; conservação filogenética de elementos de sequências; sobre-representação de sequências promotoras; temperatura de fusão de promotores; K-mer conservados; k-mer com \textit{mismatches}; \textit{k-mer median position} & SVM & rede de regulação do \textit{Saccharomyces cerevisiae}\\ \hline

\end{tabularx}



\begin{tabularx}{\textwidth}{ |X|X|X|X|X| }    \hline
		  	Trabalho      & Organismo    & Entradas     & Técnicas usadas & Resultados obtidos       \\ \hline \hline

    \cite{Lan2007}     		   & \textit{Arabidopis thaliana}  & 22.746 genes sobre 290 condições experimentais diferentes   & \textit{Logistic Regression} (LR), \textit{Linear Discriminant Analysis} (LDA), \textit{Quadratic Discriminant Analysis} (QDA), \textit{Naive Bayes} (NB) e \textit{K-Nearest Neighbors} (KNN)& Genes com grandes possibilidades de serem expressos mediante a condições de estresses abióticos \\ \hline
    \cite{Chan2005}            & \textit{Drosophila}  & CRM e não-CRM   & HexDiff & 10 novos CRMs\\ \hline
    \cite{Zhang2005}           & \textit{Arabidopsis thaliana}  & dados de plantas expostas ao ABA e/ou estresses abióticos & PWM e modelo de Markov & 1530 genes que pode ser adicionada alguma funcionalidade \\ \hline
    \cite{Bigelow2004CisOrtho} & \textit{Caenorhabditis elegans} e o \textit{Caenorhabditis briggsae}  & arquivos GFF e conjuntos de mapeamentos ortólogos  & Modelo oculto de Markov, PWM e rastros filogenéticos & mais de 25 genes alvos de TTX-3 e CEH-10  \\ \hline           

\end{tabularx}