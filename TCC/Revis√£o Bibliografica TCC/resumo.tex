\chapter{Introdução}

Com o avanço na engenharia genética, hoje é possível criar plantas resistentes a algumas pragas e a fatores adversos a sobrevivência da planta. Mas para que isto ocorra é importante o entendimento da funcionalidade de cada gene no organismo e quais são as respostas do gene a um determinado estímulo. Ainda existem muitos genes com a funcionalidade não descoberta em diversos organismos, deixando um lacuna para novas pesquisas e metodologias que visão a descoberta da funcionalidade destes genes. Ao decorrer dos anos muitas fatores de transcrição foram descobertos assim como a funcionalidades de muitos genes, com isto muitas metodologias \textit{in silico} surgiram aproveitando estas descobertas. 

O problema de encontrar a funcionalidade de genes através de dados de genes e fatores de transcrição conhecidos, remete a problemas de classificação. Os algoritmos de aprendizado de máquina são conhecidos por obterem resultados bons com dados de difícil separação e alta dimensionalidade como são os dados genômicos. Esta revisão apresenta uma breve explicação dos  fatores de transcrição com foque no \textit{dehydration responsive element binding proteins}, também são apresentado métodos computacionais que auxiliam inferir a funcionalidade de um gene.
