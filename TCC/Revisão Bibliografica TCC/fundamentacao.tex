\chapter{O papel do \textit{dehydration responsive element binding proteins} (DREB) na transcrição}

No amplo conjunto de fatores de transcrição, existem aqueles que quando ligados nos elementos regulatórios irão ativar as respostas da célula a estresses abióticos. O estresse abiótico afeta diversos organismos, mas em especial os organismos vegetais que são dependentes de fatores ambientais, são os mais afetados. O DREB faz parte do conjunto de fatores de transcrição relacionados a estímulos abióticos, e se destaca, porque ele regula genes que respondem aos estímulos abióticos de mudanças de temperatura, alta salinidade, e seca. Nas próximas seções são apresentadas as principais características relevantes para estudos computacionais, dos fatores de transcrição, elementos regulatórios e por ultimo o DREB.


\section{Fatores de transcrição}

Os fatores de transcrição (TFs, do inglês \textit{Trascription Factors}), representam uma classe especial de proteínas \cite{Werner2009}. Eles atuam no inicio da expressão de um gene, e operam na regulação da transcrição de um gene. Os TFs se conectam em pequenas regiões (5 a 20 nucleotídeos) no DNA chamadas de elementos regulatórios , quando conectados a um gene atuam na regulação do mesmo. A regulação pode ser tanto positiva (irá reforçar a transcrição), como negativa (irá inibir a transcrição). A regulação da transcrição de um gene é centralizada na expressão de um gene para um tecido específico (p. ex. um tecido de um órgão), e na regulação da ativação de um gene em resposta a um estimulo específico \cite{Latchman1997}.

Pesquisas como a de \cite{Davidson1983}, mostraram que genes que são expressos em repostas a um determinado estímulo, como elevadas temperaturas, na região promotora destes genes é possível encontrar elementos regulatórios comuns entre estes genes. Mas estes elementos regulatórios não apareciam em genes que não respondiam ao estímulo. Com esta especificidade dos elementos regulatórios, e consequentemente dos TFs como já comentado, é possível classificar TFs segundo os genes que eles regulam, de fato existem várias famílias de TFs em diversos organismos.

\section{Elementos regulatórios}

Os elementos regulatórios são pequenos seguimentos de sequência (5 a 20 nucleotídeos), localizados na região promotora\footnote{É uma região não transcrita do gene localizada antes da parte que é transcrita} de um gene, geralmente são encontrados aproximadamente a uma distância de -50 pb do sítio de início da transcrição\footnote{Ponto exato onde inicia-se a transcrição é um delimitador entre a região promotora e a parte onde vai ser transcrita}. São nos elementos regulatórios que os fatores de transcrição irão se conectar. 

Múltiplos elementos regulatórios formam os CRMs (do inglês, \textit{cis-elements modules}), que integra a conexão de vários TFs resultando em um controle combinatório, e em um padrão específico da expressão de um gene \cite{Priest2009}. Entendendo as funções dos elementos regulatórios e dos CRMs é essencial para compreender as respostas celulares ao ambiente \cite{Priest2009}.

\section{Dehydration responsive element binding proteins (DREB)}

O \textit{dehydration responsive element binding proteins} (DREB) é um importante fator de transcrição encontrado em plantas. O DREB ativa genes que estão relacionados com a resposta da célula a estímulos abióticos, com a ativação destes genes, a planta se adapta as condições adversas a sua sobrevivência, através de reações bioquímicas e físicas que ocorrem na planta. Os estresses abióticos que mais afetam as plantas são: seca, alta salinidade e mudanças de temperatura. O estresse abiótico atrapalha a sobrevivência e consequentemente a produção de grãos como a soja, arroz, milho e o trigo.

O DREB está contido dentro de uma família de fatores de transcrição única nas plantas, a \textit{Ethylene Responsive Element} (ERF). A ERF desempenha um importante papel em resposta a estímulos abióticos e bióticos. O DREB pode ser dividido em duas subclasses DREB1 e DREB2, envolvidas em estresses de baixa temperaturas e desidratação, respectivamente \cite{Chen2007GmDREB2}.

O elemento regulatório que se conecta no DREB é chamado de \textit{dehydration responsive element} DRE, e ele é formado pela sequência A/GCCGAC \cite{Nakashima2009}. Entretanto para que o DRE seja funcional ele tem que estar acompanhado por outros elementos regulatórios, mas a especificidade desses elementos regulatórios é baixa, fazendo que muitos desses varie de gene para gene \cite{Zhang2005}. Portanto, em uma busca computacional de genes alvos do DREB (ou outro fator de transcrição), não basta procurar o DRE (ou outro elemento regulatório), na região promotora, uma vez que outros elementos regulatórios também participam da regulação, isto torna a busca por genes alvo uma difícil tarefa, uma vez que ha uma variação nos elementos regulatórios de um gene para outro.

Segundo \cite{Agarwal2006}, o entendimento do DREB na regulação de um gene é de grande importância para o desenvolvimento de plantas tolerantes a estresses. Já que, estresses abióticos e bióticos influenciam negativamente na sobrevivência e na larga produção de grãos. Culturas como soja, arroz e trigo que são amplamente usadas na alimentação mundial são prejudicadas pelos estresses que muitas vezes impedem uma alta produtividade. 
