\chapter{Conclusão}

Podemos observar que a maioria dos trabalhos realizados na busca de genes alvos de fatores de transcrição, estão em comum acordo quanto a degeneração dos elementos regulatórios, e a dependência de um elemento regulatório de outros elementos regulatórios para a regulação de um gene formando os CRMs. Essas observações foram comprovadas em experimentos \textit{in vitro}, como visto em \cite{Davidson1983},\cite{Priest2009} e \cite{Zhang2005}. 

Observamos o importante papel que algoritmos de aprendizado de máquina tem na busca de genes alvos, estes algoritmos são usados em pelo menos uma das etapas das metodologias apresentadas. Em \cite{Holloway2008} e \cite{Lan2007}, é discutido que os dados genômicos apresentam uma alta dimensionalidade, o que nos leva a concluir que a predominância do SVM nas metodologias é devido a facilidade que SVM tem em lidar com alta dimensionalidades e os bons resultados obtidos mediante estas condições.

Quanto uma análise de desempenho e precisão entre os algoritmos fica inconclusiva, uma vez que os trabalhos foram feitos em diferentes tipos de organismos, mesmo os trabalhos em os organismo eram os mesmos, eles eram aplicados a diferentes tipos de fatores de transcrição.

Finalizando, é de grande importância descobrir os genes alvos de fatores de transcrição, uma vez que, ainda existem vários genes que ainda não foi associada nenhuma função. Como a função de um gene está diretamente ligada ao tipo de fatores de transcrição que conectam nele, é possível classificar estes genes a partir de fatores de transcrição conhecidos.
